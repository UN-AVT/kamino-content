\section{Learning Path}\label{basecamp-learning-path}

\subsection{Rising to the Challenge}\label{rising-to-the-challenge}

Kamino, borrowing from the Spanish word for \textit{way} or \textit{path}, is meant to be an \textit{active journey} of learning and doing.  By definition, the learning path is designed as a course to \textit{move you} to higher elevations of understanding, knowledge, and insight.  It also calls upon you to act as agents of change, to apply the skills to issues of importance for improving the world and communities we live in.

\subsection{Calling all trail blazers}\label{calling-all-trail-blazers}

A training programme specifically designed to promote the skills and capabilities to take on the toughest analytical challenges we face as staff of the United Nations. A learning experience to inspire new ways of thinking and tackling complexity. Building a community of problem solvers committed to making a difference.

\subsection{For pioneers, explorers, and innovators}\label{pioneers-innovators}

PATH to the PEAK
base camp to summit

\begin{quote}
Do not follow where the path may lead. Go instead where there is no path and leave a trail.
— Ralph Waldo Emerson
\end{quote}

\begin{quote}
Climb mountains, not so the world can see you, but so you can see the world.
— David McCullough Jr
\end{quote}

\begin{quote}
The best view comes after the hardest climb.
— Sir Edmund Hillary
\end{quote}

\begin{quote}
If you can find a path with no obstacles, it probably doesn’t lead anywhere.
— Frank A. Clark
\end{quote}

\begin{quote}
Leave the road, take the trails.
— Pythagoras
\end{quote}

\begin{itemize}
\item An Activity: a group of tasks with a similar purpose.
\item A Task: a specific objective that you need to complete.
\item A Solution: a potential method to successfully accomplish a task.
\end{itemize}

A grammar...

I need to ingest a delimited data file
Then I will wrangle the data to profile it and clean it if necessary

\begin{quote}
From a sequence of these individual patterns, whole buildings with the
character of nature will form themselves within your thoughts, as easily
as sentences. — Christopher Alexander
\end{quote}

When more than 30 years ago architect Christopher Alexander influenced the architectural world with his landmark book The Timeless Way of Building, he was introducing the idea of patterns. His thesis was that one could achieve excellence in architecture by learning and using a carefully-defined set of design rules, or patterns. This is the same principle we apply to Kamino.

The Kamino Solution (a.k.a. Pattern) Catalog --- for each activity we need to know what to do to carry out the specific tasks. This is the purpose of the catalog. The catalog is a structure to codify the instructions or solutions needed for each task, Based on a taxonomy (the learning path of Kamino), the catalog is constructed to classify a collection of
solutions called patterns.

The solution catalog is to ensure a logical structure and organization of patterns. Kamino Patterns --- the intent of patterns is to capture the essence of the practice in a compact form that can be easily communicated to those who need the knowledge. Presenting this information in a coherent, accessible, and reusable form will enable you to quickly apply pre-built templates in your own analysis projects. For every task outlined in this course, there is a corresponding solution configured to illustrate how to perform the task.

All of the solutions \sidenote{At the moment, solutions are provided for either Knime and/or R.  Contributions are welcome to expand the solution repository to include any other open source or freely available platform suitable for the purpose such as Python.} discussed are provided as workflows or code that can be re-used to execute against your own data sets.  
