\section{Introduction}\label{charts-intro}

\subsection{Size by value}\label{size-by-value}

We use the general term charts to describe the many quantitative representations for exploring and evaluating numerical data.  The common goal of these approaches is to provide visibility to structures that may be present as hidden patterns or links among seemingly unrelated items:

At their best graphics are instruments for reasoning about quantitative information. Often the most effective way to describe, explore, and summarize a set of numbers--even a very large set--is to look at pictures of those numbers. Tufte (1983: 9 )

William Playfair (1759-1823) is generally viewed as the inventor of most of the common graphical forms used to display data: line plots, bar chart and pie chart. The Commercial and Political Atlas, published in 1786, contained a number of interesting representations.

Delivered to the King, Louis XVI,
[the kind] at once understood the charts and was highly pleased. He said they spoke all languages and were very clear and easily understood. (Playfair, 1822-3)

Although the charts were novel, the King had no difficulty in grasping their purpose. (pg 1)

As knowledge increases amongst mankind, and transactions multiply, it becomes more and more desirable to abbreviate and facilitate the modes of conveying information from one person to another, and from on individual to the many. (Playfair, 4)

The advantage proposed, by this method, is not that of giving more accurate statement than by figures, but it is to give a more simple and permanent idea of the gradual progress and comparative amounts, at different periods, by presenting to the eye a figure, the proportions of which correspond with the amount of the sums intended to be expressed. (Playfair, x)

Figures and letters may express with accuracy, but they can never represent either number or space. (Playfair, xiii)

The advantages proposed by this mode of representation, are to facilitate the attainment of information, and aid the memory in retaining it: which two points form the princple benefits in what we call learning, or the acquisition of knowledge. (Playfair, 14)

The representation examples described in this chapter are best considered as recipes. Confucius wisdom “Give a man a fish and he will eat for a day. Teach a man to fish and he will eat for a lifetime.”

\begin{description}
	\item[Coordinate Systems] ...
	\item[Shapes] ...
\end{description}

Small Multiples and Facets

Small multiples is a term popularized by Edward Tufte in Envisioning Information to describe a series of small similar pictures, making a point through repetition. As Tufte writes:

At the heart of quantitative reasoning is a single question: Compared to what? Small multiple designs, multivariate and data bountiful, answer directly by visually enforcing comparisons of changes, of the diff erences among objects, of the scope of alternatives. For a wide range of problems in data presentation, small multiples are the best design solution.
(Envisioning Information, p. 67)

Small multiples are sets of thumbnail sized graphics on a single page that represent aspects of a single
phenomenon. They:
• Depict comparison, enhance dimensionality, motion, and are good for multivariate displays
• Invite comparison, contrasts, and show the scope of alternatives or range of options
• Must use the same measures and scale.
• Can represent motion through ghosting of multiple images
• Are particularly useful in computers because they often permit the actual overlay of images, and
rapid cycling.

A shared-axis multipanel is another special case of a multipanel display. A shared-axis multipanel is a multipanel display consisting of panels that share a metric axis, and that are arranged in a lineup - aligned with each other with regard to this shared metric axis.

In a shared-axis multipanel with a horizontal shared axis the panels are arranged one above the other, while in a shared axis multipanel with a vertical shared axis the panels are arranged one next to the other. In other words, thee direction of the lineup of the panels is orthogonal to the direction of their shared axis. For examples of shared-axis multipanels see the illustration of thee menstrual cycle in figure 2-46 and the illustration of the march of Napoleon’s army to Moscow in figure 2-47.